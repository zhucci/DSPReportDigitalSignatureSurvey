%Классификация методов стеганографической ЦП аудио файлов

\subsubsection{Метод внедрения цифровой подписи в аудио файлы}
\par  Методы встраивания ЦВЗ в аудио файлы можно классифицировать по области встраивания (\cite{borisova2015}) на
\begin{enumerate}
\item встраиваемые во временную область:
		\begin{itemize}
			\item метод замены наименьших значащих бит;
			\item метод внедрения информации с использованием эхо-сигнала;
			\item изменение масштаба временной (\textit{time base modulation});
		\end{itemize}
\item встраиваемые в частотную область:
		\begin{itemize}
			\item фазовое кодирование;
			\item растяжение спектра;
			\item модификация полосы частот;
			\item маскирование ЦВЗ;
		\end{itemize}
\end{enumerate} 
\par \textit{Метод замены наименьших значащих бит (НЗБ)} одинаково применяется к аудио файлам, как и к изображениям. Метод основывается на замене, в каждом значении амплитуды, наименьших значащих бит на биты сообщения. 
\par \textit{ Метод внедрения информации с использованием эхо-сигнала} Состоит в формировании звукового потока, состоящего из смеси нескольких идентичных сигналов, которые слегка отстают во времени. К параметрам эхо, несущим внедряемую
информацию, относятся: начальная амплитуда, время спада и сдвиг (время задержки между исходным сигналом и его эхо). Оригинальный сигнал смешивается с одной или несколькими точными копиями, которые слегка отстают во времени. Когда сдвиг
между оригинальным сигналом и его эхо уменьшается, ССЧ человека воспринимает эхо-сигнал как добавочный резонанс. Метод слабо устойчив к сжатию.
\par \textit {Метод фазового кодирования} предполагает встраивание информации в спектр исходного сигнала. Искомый сигнал с водяным знаком получается после обратного преобразования Фурье к модифицированному спектру.
 \par \textit {Метод растяжения спектра} добавляет псевдослучайная последовательность, представляющая собой «белый шум», модулируется сигналом несущей, представляющей ЦВЗ и затем добавляется к аудиосигналу-контейнеру.
 \par Метод \textit {изменения масштаба временной оси}  делит исходный сигнал на части, незначительно
 растягивая и сжимая  их. Местоположение и степень сжатия/растяжения являются величинами, которые используются для кодирования информации.
 
 \subsubsection{Цифровые отпечатки аудио файлов}
 \par Технология цифровых отпечатков используется в настоящее время
 для защиты различного рода информации, большей частью для защиты текстовых документов от утечек. Однако данная технология может быть распространена и на другие типы файлов, например на изображения, аудио- и видеофайлы. Она также является мощным средством удостоверения авторства на музыкальные произведение, так как процесс получения ЦО построен таким образом, что воспринимаемые одинаково звуковые образы, приводят к одинаковому отпечатку. Благодаря данному свойству, возможно определить идентичность музыкального фрагмента, перезаписанного без использования исходного файла (например, исполнение на музыкальном инструменте или с помощью синтезаторов звука).
 \par Два эффективных метода получения отпечатка файла рассмотрены в работе \cite{borisova2015}. Суть данных методов в сохранении спектральной характеристики сигнала для отдельных небольших его сегментов. Такой подход далек от совершенства, но хорошо подходит для отслеживания нелегального распространения контента в интернете, даже в сильно измененном виде.
 
