

\documentclass{article}
\textwidth 390pt
\headheight 12pt
\parindent 1.5em
\arraycolsep 5pt

\usepackage[utf8]{inputenc}
\usepackage[russian]{babel}


\begin{document}
\author{Жучков А.}
\begin{center}
Отчет о лабораторной работе \\
по курсу "Планирования последовательности разборки \\ 
технического изделия"
\end{center}
\centering{Вариант 7}

\begin{flushleft}
Нехорошо \hspace{2cm} душе без знания, и торопливый ногами оступится.\\
Кто приобретает разум, тот любит душу свою.
\end{flushleft}

\raggedleft Восточная мудрость о пользе учения.
\begin{quotation}
Цитата внезапно вставленная из википедии.
\end{quotation}
\begin{enumerate}
\item {\rule[+4mm]{10cm}{.1mm} <Первый пункт>}
\item <Последний пункт>
\end{enumerate}
\tableofcontents
\listoftables
\part{Введение}

\section*{Раздел}
\subsection*{Подраздел}
\par Абзацы отделяются друг от друга пустой строкой.

Две или три пустые строки подряд не влияют на работу \LaTeX’а.
Продолжение без отступа

\part{Основная часть}
\par В таблице \ref{t-one} вы можете видеть что-то.
\begin{table} [h]
<тело таблицы>
\caption{Стандартные размеры шрифтов в \LaTeX’е.} \label{t-one}
\end{table}

\part{Заключение}

\end{document}