\par \hspace{1.25cm}Почти сорок лет продолжается процесс развития методов формирования и проверки ЭЦП. За это время было предложено большое количество различных методов, но по-настоящему популярными стали лишь два протокола основанные на асимметричном шифровании. Криптографические ЭЦП имеют под собой как законодательную базу, так и большое количество программного обеспечения и накопленного опыта использования. Все это надежно закрепляет за данными методами место лидеров по распространению. 

\par Стеганографические методы цифровой подписи, вынуждены использовать методы шифрования для построения асимметричных схем ЦП. А значит, данные методы априори сложнее, требуют более сложных алгоритмов и сложнее программируются. Эти особенности сделали СЦП непопулярными. Хотя они и предоставляют дополнительный фактор защиты -- скрытность передачи. Последнее может в будущем стать негативным обстоятельством. В современном мире появилось дополнительное обстоятельство, которое может ограничить возможность использования скрытой передачи данных -- террористическая угроза. Криптографические методы не скрывают факта передачи защищенного сообщения, а значит поддаются контролю со стороны государственных специальных служб. В этой ситуации стеганографические методы передачи данных становятся объектом пристального внимания террористов.Также относительная редкость применения стеганографических методов, особенно в финансовой сфере,  обуславливает малую выгоду от разработки алгоритмов взлома и методов атак, делая потенциальные атаки слишком затратными. 
\par Последнее десятилетие на смену электронной коммерции приходит мобильная. Мобильные устройства отличаются малыми вычислительными мощностями. Этот фактор ограничивает уровень надежности алгоритмов шифрования, которые можно использовать для проведения электронных платежей. В данных условиях может стать целесообразным и экономически обоснованным разработка и поддержка уникального стеганографического метода шифрования и формирования ЦП для удостоверения платежей на мобильных устройствах. 
\par Надежность цифровой криптографической подписи основывается на математических  задачах, сложность которых настолько велика, что они не могут быть решены за время жизни самой подписи. У данного подхода есть две существенные проблемы. Во-первых, сложность этих задач не доказана, а значит надежность не гарантированная. Во-вторых, хоть и в очень отдаленной перспективе, производительность компьютеров может значительно возрасти с созданием рабочей версии квантового компьютера. Его возможности позволят решить все задачи, относящиеся к классу \textit{np}-полных за полиномиальное время. Это приведет к невозможности использовать традиционные асимметричные алгоритмы формирования ЦВЗ. Конечно, для решения этой проблемы были разработаны специальные квантовые алгоритмы цифровой подписи. В работе подробно рассмотрены алгоритмы КЦП электронных документов \cite{yin2015}. Но было бы интересно провести анализ гибридных крипто- стеганографических алгоритмов на стойкость к атаке с применением квантового компьютера.
\par  