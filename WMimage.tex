\subsubsection{Метод внедрения цифровой подписи в изображения}
\par В настоящее время предложено множество методов встраивания информации в изображение. Обзор методов встраивания ЦВЗ в изображения представлен в работе \cite{sagaydak2014}. Разработаны такие пространственные методы, как:
\begin{enumerate}
\item метод модификации младших бит \textit{LSB (Last Significant Bit)};
\item метод случайного интервала;
\item метод псевдослучайной перестановки (выбора);
\item метод блочного сокрытия.
\end{enumerate}
\par В отличие от метода LSB, в котором каждый бит
скрываемого сообщения записывается в последовательно идущие младшие биты, метод случайного интервала позволяет осуществлять случайное распределение битов этого сообщения по контейнеру, в результате чего расстояние между двумя встроенными
битами скрываемого сообщения определяется случайным образом. Но есть и недостаток данного метода -- биты скрываемого сообщения в контейнере
размещаются в той же последовательности, что и в
самом скрываемом сообщении. Поэтому, во избежание этого недостатка, прибегают к методу псевдослучайной перестановки (выбора), суть которого заключается в том, что при помощи генератора псевдослучайных чисел образуется последовательность индексов $j_1,j_2,\dots,j_k$ и выполняется сохранение $k$ -го бита
сообщения в пикселе с индексом $j_k$ .
Суть метода блочного скрытия заключается в следующем: изображение-оригинал разбивается на $l_m$
непересекающихся блоков $ \Delta i ( 1 \le i \le l_m ) $ произвольной конфигурации, для каждого из которых вычисляется бит чётности $b ( \Delta i ) = \sum_{j \in \Delta i}^{mod 2} LSB(C_j)$.
В каждом блоке выполняется скрытие одного секретного бита
$M_i$ . Если бит чётности $b ( \Delta i ) \ne M_i$ , то происходит
инвертирование одного из наименьших значащих битов блока $\Delta i$ , в результате чего $b ( \Delta i ) = M_i$ . Выбор блока может происходить псевдослучайно с использованием стеганоключа.

Также в своей работе Сагайдак Д. представил новый метод, позволяющий производить скрытое встраивание ЦВЗ как в изображение, так и физический документ. Идея метода встраивания основывается также на методе \textit{LSB}. 
Технически методика встраивания не меняется, но предложенный метод позволяет адаптировать ЦВЗ к конкретному документу, встраивая данные лишь в незначащие пиксели. Данный ЦВЗ относится к видимым. Но все же может применяться для подписи и удостоверения документа, если выполняется предположение о секретности ключа. Также в статье большое внимание уделено сложности легитимного распознавания наличия водяного знака и расшифровки информации содержащейся в нем. 