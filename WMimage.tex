\subsubsection{Метод внедрения цифровой подписи в изображения}
\par В настоящее время предложено множество методов встраивания информации в изображение. Обзор методов встраивания ЦВЗ в изображения представлен в работе \cite{sagaydak2014}. Разработаны такие пространственные методы, как:
\begin{enumerate}
\item метод модификации младших бит \textit{LSB (Last Significant Bit)};
\item метод случайного интервала;
\item метод псевдослучайной перестановки (выбора);
\item метод блочного сокрытия.
\end{enumerate}
Также в своей работе Сагайдак Д. представил новый метод, позволяющий производить скрытое встраивание ЦВЗ как в изображение, так и физический документ. Идея метода встраивания основывается на методе LSB. В статье большое внимание уделено сложности легитимного распознавания наличия водяного знака и расшифровки информации содержащейся в нем. 