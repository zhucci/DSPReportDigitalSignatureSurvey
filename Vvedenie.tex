\par \hspace{1.25cm}Рукописная подпись используется для удостоверение подлинности
 документа, а также авторства или согласия с содержимым. С масштабным переходом
  к системам электронного документооборота (СЭД), остро встала проблемы
   проверки аутентичности данных и установления авторства электронного
    документа. Для решения данных проблем были разработаны методы электронной подписи документов. Электронная цифровая подпись (ЭЦП) --- особый атрибут электронного документа, хранящийся вмести с ним. Данный атрибут позволяет защитить данные от изменения и однозначно устанавливает связь межу электронной подписью и лицом (группой лиц), подписавшим данный документ.
\par Для мультимедиа файлов, охраняемых авторским правом, также актуальны
 проблемы целостности и аутентификации автора при их передаче, но
  появляется дополнительная задача – пресечение попыток незаконного
   распространения. Данная задача решается, созданием устойчивой электронной
    подписи. Устойчивая ЭЦП (устойчивый водяной знак) -- атрибут мультимедиа файла, хранящий информацию о субъекте авторского права, который нельзя уничтожить обычными манипуляциями с данными. 
\par Такая задача, как определение целостности (подлинности) документов, в
 настоящее время поднимается всё чаще. Это вызвано увеличением объёма документооборота между организациями, а также развитием технологий обмена документами. В связи с этим появляется множество различных методов защиты документов от подделки. Так, возможно применение криптографических методов цифровой подписи (ЭЦП), стойкость которой основывается на сложности вычисления дискретного логарифма в группе точек эллиптической кривой, а также на стойкости хэш-функции \cite{gost34.10}, или стеганографических методов \cite{gupta2015} – встраивания ЦП в электронный документ (изображение) или встраивания ЦВЗ, а также внесения специального шума. 