%Классификация методов стеганографической ЦП аудио файлов
\documentclass{report}
\usepackage[english,russian]{babel}
\usepackage[utf8]{inputenc}
\usepackage{csquotes}
\usepackage[style=authoryear,backend=biber]{biblatex}
\addbibresource{Bibliography.bib}
\begin{document}

\subsection{ЦВЗ аудио файлов}
\par  Методы встраивания ЦВЗ в аудио файлы можно классифицировать по области встраивания \cite{борисова2015} на
\begin{enumerate}
\item встраиваемые во временную область
		\begin{enumerate}
			\item метод замены наименьших значащих бит;
			\item метод внедрения информации с использованием эхо-сигнала;
			\item изменение масштаба временной оси (time base modulation);
		\end{enumerate}
\item встраиваемые в частотную область
		\begin{enumerate}
			\item Фазовое кодирование
			\item Растяжение спектра
			\item Модификация полосы частот
			\item Маскирование ЦВЗ
		\end{enumerate}
\end{enumerate} 
\par \it {Метод замены наименьших значащих бит (НЗБ)} одинаково применяется к аудио файлам, как и к изображениям. Метод основывается на замене, в каждом значении амплитуды, наименьших значащих бит на биты сообщения. 
\par \it {Метод внедрения информации с использованием эхо-сигнала} Состоит в формировании звукового потока, состоящего из смеси нескольких идентичных сигналов, которые слегка отстают во времени. К параметрам эхо, несущим внедряемую
информацию, относятся: начальная амплитуда, время спада и сдвиг (время задержки между исходным сигналом и его эхо). Оригинальный сигнал смешивается с одной или несколькими точными копиями, которые слегка отстают во времени. Когда сдвиг
между оригинальным сигналом и его эхо уменьшается, ССЧ человека воспринимает эхо-сигнал как добавочный резонанс. Метод слабо устойчив к сжатию.
\par \it {Метод фазового кодирования} предполагает встраивание информации в спектр исходного сигнала. Искомый сигнал с водяным знаком получается после обратного преобразования Фурье к модифицированному спектру.
 \par \it {Метод растяжения спектра} добавляет псевдослучайная последовательность, представляющая собой «белый шум», модулируется сигналом несущей, представляющей ЦВЗ и затем добавляется к аудиосигналу-контейнеру.
 \par \it Метод {изменения масштаба временной оси}  делит исходный сигнал на части, незначительно
 растягивая и сжимая  их. Местоположение и степень сжатия/растяжения являются величинами, которые используются для кодирования информации.
 
 \subsection{Цифровые отпечатки аудио файлов}
 \par Цифровой отпечаток --- перспективная технология, не относящаяся к водяным знакам, однако предоставляющая мощное средство удостоверения авторства на музыкальное произведение. ЦО 
 

\end{document}
